% Fixed by fix.awk 2015-03-09 12:06:49
%%!TEX TS-program = dtxmake
%%!TEX dtxmake-subengine = xelatexmk
%<*driver>
\newwrite\configfile
\def\writeconfig{\begingroup\setwccatcodes\innerwriteconfig}
\def\setwccatcodes{\catcode`\#=12 \catcode`\%=12 }
\def\innerwriteconfig#1{\immediate\write\configfile{\unexpanded{#1}}\endgroup}
\immediate\openout\configfile=bundledoc.cfg
\writeconfig{
bundle: (pdftk $BDBASE.pdf attach_files $BDINPUTS/* output $BDBASE.bd.pdf)^^J
find:   /usr/texbin/kpsewhich -progname=latex $BDINPUTS^^J
}
\immediate\closeout\configfile
\immediate\openout\configfile=.latexmkrc
\immediate\write\configfile{%
push @generated_exts, "ttl";^^J
push @generated_exts, "bbl";^^J
push @generated_exts, "fls";^^J
push @generated_exts, "synctex.gz";^^J
push @generated_exts, "dep";^^J
push @generated_exts, "md5";^^J
push @generated_exts, "tex.md5";^^J
}
\immediate\closeout\configfile
% archive script
% create a blank evernote and get its URL first
\def\evernoteURL{}
\immediate\openout\configfile=archive.sh
\writeconfig{
#!/bin/sh -v^^J
for file in `ls *.dep`; do /usr/texbin/bundledoc --verbose --config=bundledoc.cfg --include=*.dtx $file; done^^J
latexmk -c^^J
rm -f *.bbl *.log *.{ws,ws-sol,lp}.tex
}
\immediate\closeout\configfile
\immediate\openout\configfile=config.def
\immediate\write\configfile{\string\def\string\rootjobname{\jobname}}
\immediate\write\configfile{\string\def\string\evernoteURL{\evernoteURL}}
\immediate\closeout\configfile
\input docstrip.tex
\askforoverwritefalse
\generate{
    \file{\jobname.lp.tex}{\from{\jobname.dtx}{lesson,plan}}          
    \file{\jobname.ws.tex}{\from{\jobname.dtx}{worksheet}}
    \file{\jobname.ws-sol.tex}{\from{\jobname.dtx}{worksheet,solutions}}
}
\endbatchfile
%</driver>
\documentclass
%<slides|handout>[ignorenonframetext,aspectratio=169,14pt]
%<slides>{ngelessonslides}
%<handout>{ngelessonhandout}
%<plan>{ngelessonplan}
%<worksheet&solutions>[solutions]
%<worksheet>{ngelessonworksheet}
%%
\title[
%	short title=,
	lesson number=16,
	textbook section=11.7
	]{Maximum and Minimum Values}
\input{../course.def}
\usepackage[us]{datetime}
\newdate{lessondate}{8}{3}{2016}
\edef\lessondateiso{\getdateyear{lessondate}-\twodigit{\getdatemonth{lessondate}}-\twodigit{\getdateday{lessondate}}}
\date{\displaydate{lessondate}}
\author{Professor Matthew Leingang}
\institute[NYU]{New York University}
\makeatletter
\keyword{course: \ngelesson@course@code\space\insertcoursename}
\keyword{term: \ngelesson@term}
\keyword{STEC\ngelesson@textbooksection}% must fix if chap > 9
\keyword{\insertcoursename}
\makeatother
\keywords{Fermat's Theorem, maximum value, minimum value}
\lessondescription{A central application of multi-variable calculus is to extremize
functions of two variables. One first identifies critical points,
points where the gradient vanishes. The nature of these critical points
can be established using the second derivative test. There will 
be three fundamentally different cases: local maxima, local minima 
as well as saddle points.}% Oliver's description
\input config.def
\usepackage{siunitx}
\usepackage{overpic}
\usepackage{snapshot}
\usepackage{xr}
\usepackage{zref-xr,zref-user}
\usepackage{textcomp}
\externaldocument{\rootjobname.ws}
%<*lesson>
\usetheme
%<OpenEd>[margins]
{NYUOpenEd}
%\useoutertheme{infolines}% to test metadata setting
%<*plan>
\usepackage[no-math]{fontspec}
\usepackage{xltxtra}
\defaultfontfeatures{Mapping=tex-text}
\setmainfont{Cambria}
%\setmathrm{Cambria Math}
\setsansfont{Calibri}
%</plan>

\usepackage{leincalc}
\usepackage{tikz}
\usetikzlibrary{MATH-UA-123}



\newcommand{\newVector}[2]{
	\newcommand{#1}{\Vector{#2}}
}
\newVector{\va}{a}
\newVector{\vb}{b}
\newVector{\vc}{c}
\newVector{\vF}{F}
\newVector{\vr}{r}
\newVector{\vT}{T}
\newVector{\vn}{n}
\newcommand{\dr}{d\vr}
\newcommand{\dS}{d\Vector{S}}

\usepackage{hyperref}
\makeatletter
\hypersetup{colorlinks=true,
citecolor=tertiary,
filecolor=primary,
urlcolor=primary,
pdftitle=\ngelesson@longtitle,
pdfauthor=\insertauthor,
pdfsubject=\insertcoursename,
pdfkeywords=\the\ngelesson@keywords,
}
\makeatother

\usepackage{datatool}
\DTLloaddb{announcements}{../announcements.csv}
\dtlsort{endDate}{announcements}{\dtlcompare}
\begin{document}

\begin{frame}[plain]
\maketitle

\section*{Announcements}
\timecheck{\LARGE\clock{2}{00}}
\begin{itemize}
\DTLforeach[\DTLisgt{\endDate}{\lessondateiso}\and\not\DTLisgt{\startDate}{\lessondateiso}]{announcements}{\annText=announcementText,\endDate=endDate,\startDate=startDate}{\item \annText}
\end{itemize}
%\mode<presentation>{\photocredit{}{}}
\end{frame}


%\mode<all>{\egroup}

\begin{tikzpicture}[remember picture,overlay]
\node[anchor=north west] at ([xshift=0.5in,yshift=-0.5in]current page.north west) 
{\includegraphics[width=2.5in]{NYUCourant}};
\end{tikzpicture}





\mode<article>{%
\section*{Objectives}
}% end mode<article>

\begin{frame}{Objectives}
\begin{columns}
\begin{column}{0.45\textwidth}
\begin{itemize}
\item Find the critical points of a function of several variables.
\item Classify the critical points of a function of two variables.
\end{itemize}\end{column}
\begin{column}{0.45\textwidth}
\mode<presentation>{\includegraphics[width=\textwidth]{Anonymous_target_with_arrow}}
\end{column}
\end{columns}
\end{frame}


\subsection*{Notes}

\makeatletter
See also \cite[Section \ngelesson@textbooksection]{Stewart-EC}.
\makeatother

Oliver's summary: A central application of multi-variable calculus is to \textbf{extremize}
functions of two variables. One first identifies \textbf{critical points}, 
points where the gradient vanishes. The nature of these critical points
can be established using the \textbf{second derivative test}. There will 
be three fundamentally different cases: \textbf{local maxima}, \textbf{local minima} 
as well as \textbf{saddle points}. 

\subsubsection{Fall 2015}

After class, I rewrote the section deriving the quadratic approximation.  This should tighten it up a bit.
Otherwise, don't add anything.  With the Do Now/Anticipatory Set this lesson will fill up 1:50 easily.

\subsubsection{Spring 2015}

Blind copy (I guess).

\subsubsection{Fall 2014}

Blind copy.

\subsubsection{Fall 2013}

See \href{}{notes in Evernote \includegraphics{Icon_External_Link.pdf}}.


\section*{Materials}

\begin{itemize}
\item colored chalk/markers
\end{itemize}


\mode<article>{
\section*{Do Now}}


Let $F(x,y,z) = z-x^2-y^2$.  Find $\nabla f$.
Sketch the level surface $F=0$ and draw $\nabla f$ at various points $P$ attached to $P$:
$(0,0,0)$, $(\pm1,0,1)$, $(0,\pm1,1)$.




\mode<article>{%
\section*{Anticipatory Set}}
\timecheck{\LARGE\clock{9}{10}}

One of the most important problems in calculus is finding the maximum and minimum values of a function.
In one-variable calculus, we learned to take derivatives to find critical points, and something called the \emph{second derivative test} for determining whether a critical point was positive or negative.

You may want to review Sections 4.1 and 4.3 to refresh yourself on this.

\begin{example}
Let $f(x) = x^3 - 3x$.  Find and classify the local maxima and minima.
\begin{solution}
\emph{Fermat's Theorem} says that the local extrema occur where $f'(x) =0$.
We have
\[
	f'(x) = 3x^2 - 3 = 3 (x+1)(x-1)
\]
So the critical points are $x = \pm 1$.  

The \emph{Second Derivative Test} says the sign of the second derivative at a critical point can be used to classify that critical point.  We have:
\[
	f''(x) = 6x.
\]
So $f''(1) = 6 > 0$, meaning this critical point is a minimum.  Near that point, the graph of the function looks like $y=3x^2$.  Hence, a minimum.  OTOH, $f''(-1) < 0$, so this point is a local maximum.
\end{solution}
\end{example}

Why does this work?
The linear approximation of a one-variable function $f(x)$ near $a$ is
\[
	f(a) \approx f(a) + f'(a) h
\]
and if we take it one step further we may say
\[
	f(a+h) \approx f(a) + f'(a) h + \tfrac12 f''(a) h^2
\]
(The error is equal to $\frac{1}{6}f'''(z)h^3$ for some $z$ between $0$ and $h$.
So if $h$ is small and $f'''$ is bounded the error is small.  This is from Taylor's theorem)
This approximation illustrates the second derivative test for one-variable functions.
If $f'(a) = 0$ and $f''(a) > 0$, then the graph of $f$ looks like an upward opening parabola with a vertex at $(a,f(a))$.
\Do{sketch}

We want to generalize this to higher dimensions.  
\begin{itemize}
\item What is the analog of the condition that $f'(a) = 0$?  
\item What is the analog of the second derivative?
\item What is the analog of the second derivative test?
\end{itemize}



\tableofcontents


\mode<article>{%
\section*{Procedure}}
\section{Critical Points}

\begin{definition}
A function of two variables has a \defined{local maximum} at $(a,b)$ if $f(x,y) \leq f(a,b)$ whenever $(x,y)$ is near $(a,b)$.  [This means that $f(x,y) \leq f(a,b)$ for all points $(x,y)$ in some disk with center $(a,b)$.]  The number $f(a,b)$ is called a \defined{local maximum value}.  If $f(x,y) \geq f(a,b)$ when $(x,y)$ is near $(a,b)$, then $f(a,b)$ is called a \defined{local minimum value}.
\end{definition}

If the inequalities hold for all $(x,y)$ in the domain of $f$, then $f$ has an \defined{absolute maximum} (or \defined{absolute minimum}) at $(a,b)$.

\begin{theorem}[Fermat's Theorem]
If $f$ has a local maximum or minimum at $(a,b)$, and the first-order partial derivatives of $f$ exist there, then $f_x(a,b) = 0$ and $f_y(a,b) = 0$.
\end{theorem}

\begin{proof}
The one-variable function $g(x) = f(x,b)$ has a local maximum (or minimum) at $x=a$, so by the ordinary Fermat's theorem, $g'(a) = 0$.  But $g'(a) = f_x(a,b)$.  
We can do the same thing for $y \mapsto f(a,y)$.
\end{proof}

We call $(a,b)$ a \defined{critical point} of $f$ if $f_x(a,b) = 0$ and $f_y(a,b) = 0$.
In more vector terms, we say $\nabla{f}(a,b) = \Vector{0}$.
This makes sense in terms of the geometric significance of the gradient: $\nabla f$ points in the direction of greatest increase of $f$.
If $\nabla{f}(a,b) = \Vector{0}$, there is no direction that will increase $f$.

\begin{example}
\label{example:1}
Let $f(x,y) = 6x^2 - 2x^3 + 3y^2 + 6xy$.
Find the critical points.
\begin{solution}
\begin{align*}
	\frac{\partial f}{\partial x}(x,y)
		&= 12x - 6x^2 + 6y  \\
	\frac{\partial f}{\partial x}(x,y)
		&= 6y+6x  \\
\end{align*}
If $6y+6x =0$, then $y=-x$, and substituting into the first one we get
\[
	12x - 6x ^2 - 6x = 6x - 6x^2 = 6x(1-x).
\]
So the critical points are $x=0$ (hence $y=0$) and $x=1$ (hence $y=-1$).
\end{solution}
\end{example}

\begin{practice}% E&P, Example 12.5.3
Find the critical points of the function
\[
	f(x,y) = \frac{3}{4}y^2 + \frac{1}{24}y^3 - \frac{1}{32}y^4 -x^2
\]
\begin{solution}
We have
\begin{align*}
	\pderiv{f}{x} &= -2x  \\
	\pderiv{f}{y} &= \frac{3}{2}y + \frac{1}{8} y^2 - \frac{1}{8} y^3 \\
	&= -\frac{1}{8}y(y^2-y-12) = -\frac{1}{8}y(y+3)(y-4)
\end{align*}
Setting $\pderiv{f}{x}=0$ and $\pderiv{f}{y} = 0$ gives $x=0$, and $y=0$, $-3$, or $4$.
So there are three critical points: $(0,0)$, $(0,-3)$, $(0,4)$.
\end{solution}
\end{practice}

\begin{practice}% E&P Exercise 12.5.19
Find the critical points of the function
\[
f(x,y) = 2x^2 + 8xy + y^4
\]
\begin{solution}
The critical points are $(0,0)$, $(4,-2)$, and $(-4,2)$.
\end{solution}
\end{practice}

\Do{Perfect place for a break}

\section{In Search of a Second Derivative Test}
\subsection{Quadratic forms and the Hessian}
%\timecheck{\LARGE\clock{11}{15}}

\marginpar{\dbend}

The problem of twice-differentiation is that it seems somewhat incomplete.  We need to look at more than just the second derivatives
$\frac{\partial^2f }{\partial x^2}$ and $\frac{\partial^2f }{\partial y^2}$.  After all, each $\frac{\partial f}{\partial x_i}$ is a function of two variables, and so second derivatives of $f$ should include all derivatives of each partial derivative.  This means there are $4$ partial derivatives.

\begin{fact}
    Let $f$ have all third-order derivatives continuous near $(a,b)$.
    Then for small $h$ and $k$,
    \begin{multline}
        f(a+h,b+k) \approx
            f(a,b) 
            + f_{x}(a,b)h + f_{y}(a,b)k \\
            + \frac12f_{xx}(a,b)h^2 + f_{xy}(a,b)hk + \frac12f_{yy}(a,b)k^2.
    \label{eqn:quad-approx-2var}
    \end{multline}
\end{fact}
\begin{proof}
    Let
    \(
        g(t) = f(a+th,b+tk)
    \).
    Then by our one-variable Taylor's theorem:
    \begin{equation}
        g(1) \approx g(0) + g'(0)(1) + \frac{1}{2} g''(0)(1)^2
        \label{eqn:quad-approx-1var}
    \end{equation}
    Now 
    \begin{equation}
        g(1) = f(a+h,b+k)\text{ and } g(0) = f(a,b)
        \label{eqn:quad-approx-2var!0o}
    \end{equation}
    Also,
    \[\begin{split}
         g'(t) &= \pderiv{f}{x}\D{x}{t} + \pderiv{f}{y}\D{y}{t} \\
         &= \pderiv{f}{x}(a+th,b+tk) h + \pderiv{f}{y}(a+th,b+tk) k \\
    \end{split}\]
    Therefore 
    \begin{equation}
        g'(0) = f_x(a,b)h + f_y(a,b) k
        \label{eqn:quad-approx-2var!1o}
    \end{equation}
    
    Furthermore,
    \mathtoolsset{firstline-afterskip=5em}%
    \[\begin{split}%
        g''(t) &\begin{multlined}[t]= \pderiv{}{x}\left(\pderiv{f}{x}(a+th,b+tk) h + \pderiv{f}{y}(a+th,b+tk) k\right)\D{x}{t} \\
                                    + \pderiv{}{y}\left(\pderiv{f}{x}(a+th,b+tk) h + \pderiv{f}{y}(a+th,b+tk) k\right)\D{y}{t}
                \end{multlined} \\
               &\begin{multlined}[t]= \pderiv[2]{f}{x}(a+th,b+tk)h^2 + \pderiv{f}{x,y}(a+th,b+tk)kh \\
                                    + \pderiv{f}{y,x}(a+th,b+tk)hk + \pderiv[2]{f}{y}(a+th,b+tk) k^2 
                \end{multlined}
    \end{split}\]
    Therefore 
    \begin{equation}
        g''(0) = f_{xx}(a,b)h^2 + 2 f_{xy}(a,b) hk + f_{yy}(a,b)k^2
        \label{eqn:quad-approx-2var!2o}
    \end{equation}
    Substituting \eqref{eqn:quad-approx-2var!0o}, \eqref{eqn:quad-approx-2var!1o}, and \eqref{eqn:quad-approx-2var!2o} into \eqref{eqn:quad-approx-1var} gives \eqref{eqn:quad-approx-2var}.
\end{proof}

This is getting complicated.  
One way to alleviate the notation and get a handle on the geometry of the situation is to express this approximation in terms of vector operations.  
For instance, the term $f_{x}(a,b)h + f_{y}(a,b)k$ is the gradient of $f$ at $(a,b)$ dotted with the vector $\Veclist{h,k}$.
What kind of operation is the quadratic term?

\begin{definition}
A \defined{quadratic form} is a second-degree polynomial with only quadratic terms (no linear or constant terms).
\end{definition}

A quadratic form in two variables looks like
\[
	Q(x,y) = ax^2 + 2bxy + cy^2
\]
When thinking in terms of vectors instead of variables, the coefficients of $Q$ can be arranged in a $2\times 2$ matrix
$\begin{bmatrix} a & b \\ b & c \end{bmatrix}$.  Then
\[
\begin{split}
	\Veclist{x,y}\begin{bmatrix} a & b \\ b & c \end{bmatrix} \cdot \Veclist{x,y}
	&= \Veclist{ax + by,bx + cy}\cdot\Veclist{x,y} \\
	&= ax^2 + bxy + byx + cy^2 = Q(x,y)
\end{split}
\]

A function of two variables gives rise to a quadratic form at each point of the domain.
\begin{definition}
Let $f(x,y)$ have all second partial derivatives at $(a,b)$.  
We define the \defined{Hessian} $\Hessian{f}(\Vector{a})$ to be the matrix
\[
		\Hessian{f}(a,b)= \begin{bmatrix} f_{xx}(a,b) & f_{xy}(a,b) \\ f_{yx}(a,b) & f_{yy}(a,b) \end{bmatrix}.
\]
\end{definition}

Thus we can express our quadratic approximation as
\[
	f(a+h,b+k) \approx
		f(a,b) + \nabla f(a,b) \cdot \Veclist{h,k}
			+ \tfrac12 \Veclist{h,k} \Hessian{f}(a,b)\cdot \Veclist{h,k}
\]
To summarize: The derivative of a function of several variables at a point is more naturally a $n$-dimensional vector $\nabla f$.  
The second derivative of a function of several variables at a point is most naturally a quadratic form, or a $n\times n$ matrix $\Hessian{f}$.  In order to understand the geometry of the graph near a critical point, we need to know more about these quadratic forms.

\subsection{Classification of Two-dimensional Quadratic Forms}
\timecheck{\LARGE\clock{11}{30}}

\Say{Let us look at some examples of quadratic forms, including their matrices, and see what the entries in the matrix can tell us about the shape of their graphs.}

\Do{Prepare the board for four examples all to be seen at once} 

\begin{example}
Let
\(
	Q(x,y) = -x^2 - y^2
\).
Then 
\(
	\Hessian{Q} = \begin{bmatrix} -2 & 0 \\ 0 & -2 \end{bmatrix}
\).
The graph of $Q$ is a downward-opening circular paraboloid, and the point $(0,0)$ is the maximum point.
\Do{draw the graph and contour plot side-by-side.}
\end{example}

\begin{example}
Let 
\(
	Q(x,y) = x^2 + y^2,
\).
Then 
\(
	\Hessian{Q} = \begin{bmatrix} 2 & 0 \\ 0 & 2 \end{bmatrix}
\).
The graph of $Q$ is an upward-opening circular paraboloid, and the point $(0,0)$ is the minimum point.
\Do{draw the graph and contour plot side-by-side.}
The function $Q$ has a minimum at $(0,0)$.
\end{example}

\begin{example}
Let 
\(
	Q(x,y) = x^2 - y^2,
\)
Then 
\(
	\Hessian{Q} = \begin{bmatrix} 2 & 0 \\ 0 & -2 \end{bmatrix}
\).
The graph of $Q$ is an hyperbolic paraboloid, and the point $(0,0)$ is the saddle point.
\Do{draw the graph and contour plot side-by-side.}
The function $Q$ has a saddle point at $(0,0)$.
\end{example}

\begin{example}
Let 
\(
	Q(x,y) = 2xy,
\)
Then 
\(
	\Hessian{Q} = \begin{bmatrix} 0 & 2 \\ 2 & 0 \end{bmatrix}
\).
The graph of $Q$ is also hyperbolic paraboloid, and the point $(0,0)$ is the saddle point.
\Do{draw the graph and contour plot side-by-side.}
The function $Q$ has a saddle point at $(0,0)$.
\end{example}

How can we tell based on the entries of the matrix what the graph of the associated quadratic form looks like?  
Algebra to the rescue.  (See Shifrin, Corollary 3.4)
Suppose $a \neq 0$.
\[
\begin{split}
	Q (x,y)
	&= ax^2 + 2bxy + cy^2 \\
	&= a\left(x^2 + \frac{2by}{a}x\right) + cy^2 \\
	&= a\left(x^2 + \frac{2by}{a}x + \left(\frac{by}{a}\right)^2\right) + cy^2 - \frac{b^2y^2}{a} \\
	&= a\left(x + \frac{b}{a} y\right)^2 + \left( c- \frac{b^2}{a}\right)y^2 \\
	&= a\left(x + \frac{b}{a} y\right)^2 + \left(\frac{ac - b^2}{a}\right)y^2 \\
\end{split}
\]
If both the coefficients of the squared terms are positive, the graph is an upward-opening elliptic paraboloid and $(0,0)$ is the minimum.  This happens when $a > 0$ and $ac - b^2 > 0$.
If both the coefficients of the squared terms are negative, the graph is a downward-opening elliptic paraboloid and $(0,0)$ is the maximum.  This happens when $a < 0$ and $ac-b^2 > 0$.
If the coefficients if the squared terms have opposite signs, the graph is a hyperbolic paraboloid and $(0,0)$ is the saddle point.

Other cases? If $a=0$ and $b,c\neq 0$, we can do the same kind of complete the square operation to get
\[
	Q(x,y) = c \left(y + \frac{bx}{c}\right)^2 - \frac{b^2}{c} x^2
\]
Since $b^2>0$, the coefficients $c$ and $-\frac{b^2}{c}$ will have opposite signs, hence the graph is has a saddle point.

If $a=c=0$ and $b \neq 0$, then $Q(x,y) = 2bxy$, and we saw before that the graph has a saddle point.

All of these cases can be summarized by:

\begin{proposition}
Given a quadratic form 
\[
	Q(x,y) = ax^2 + 2bxy + cy^2 = \Veclist{x,y} \begin{bmatrix} a & b \\ b & c \end{bmatrix} \Veclist{x,y}
\]
\begin{itemize}
\item If $ac-b^2 > 0$ and $a>0$, then $(0,0)$ is the minimum of $Q$.
\item If $ac-b^2 > 0$ and $a<0$, then $(0,0)$ is the maximum of $Q$.
\item If $ac-b^2 < 0$, then $(0,0)$ is the saddle point of $Q$.
\end{itemize}
\end{proposition}

The expression $ac-b^2$ should be familiar to some: it is the \defined{determinant} of the matrix.

\subsection{The second derivative test}
\timecheck{\LARGE\clock{11}{45}}

We are finally ready to take what we learned about quadratic forms and assemble it into the second derivative test.

\begin{theorem}[The Second Derivative Test]
Suppose the second partial derivatives of $f$ are continuous on a disk with center $(a,b)$
and that $(a,b)$ is a critical point of $f$.
Let
\[
	D = \pderiv[2]{f}{x}(a,b)\pderiv[2]{f}{y}(a,b) - \left(\pderiv{f}{x,y}(a,b)\right)^2
\]
\begin{enumerate}
\item If $D>0$ and $\frac{\partial^2 f}{\partial x^2}(a,b) > 0$, then $f(a,b)$ is a local minimum value.
\item If $D>0$ and $\frac{\partial^2 f}{\partial x^2}(a,b) < 0$, then  $f(a,b)$ is a local maximum value.
\item If $D< 0$, then $f(a,b)$ is neither a local maximum nor a local minimum.  The graph of $f$ has a saddle point at $(a,b)$.
\end{enumerate}
\end{theorem}

\section{Examples}

\begin{example}
Return to Example~\ref{example:1}.  
Let $f(x,y) = 6x^2 - 2x^3 + 3y^2 + 6xy$.
Classify the critical points.
\begin{solution}
\begin{align*}
	\frac{\partial f}{\partial x}(x,y)
		&= 12x - 6x^2 + 6y  \\
	\frac{\partial f}{\partial x}(x,y)
		&= 6y+6x  \\
\end{align*}
The critical points are $(0,0)$ and $(1,-1)$.
\begin{align*}
	\frac{\partial^2 f}{\partial x^2}(x,y) 
		&= 12 - 12x  \\
	\frac{\partial^2 f}{\partial y \partial x}(x,y) 
		&= 6 \\
	\frac{\partial^2 f}{\partial x \partial y}(x,y) 
		&=  6 \\
	\frac{\partial^2 f}{\partial y^2}(x,y) 
		&=  6 \\		
\end{align*}
We have
\[
	\Hessian{f}(0,0)
		=\begin{bmatrix} 12 & 6 \\ 6 & 6 \end{bmatrix}.
\]
The determinant is positive and the top-left is positive.
So the graph has a minimum at this point.
On the other hand,
\[
	\Hessian{f}(1,-1)		=\begin{bmatrix} 0 & 6 \\ 6 & 6 \end{bmatrix}.
\]
The determinant is negative, so the graph has a saddle point.

\begin{center}
\includegraphics[width=0.8\textwidth]{2005-10-31-Example_1.eps}
\end{center}
\end{solution}
\end{example}

\begin{practice}% E&P, Example 12.5.3
Classify the critical points of the function
\[
	f(x,y) = \frac{3}{4}y^2 + \frac{1}{24}y^3 - \frac{1}{32}y^4 -x^2
\]
\begin{solution}
We have
\begin{align*}
	\pderiv{f}{x} &= -2x  \\
	\pderiv{f}{y} &= \frac{3}{2}y + \frac{1}{8} y^2 - \frac{1}{8} y^3 \\
	&= -\frac{1}{8}y(y^2-y-12) = -\frac{1}{8}y(y+3)(y-4)
\end{align*}
There are three critical points: $(0,0)$, $(0,-3)$, $(0,4)$.
Now
\begin{align*}
	\pderiv[2]{f}{x} &= -2 & \pderiv{f}{y,x} &= 0 \\
	\pderiv{f}{x,y} &=0    & \pderiv[2]{f}{y} &= \frac{3}{2} + \frac{1}{4} y - \frac{3}{8} y^2 \\
\end{align*}
The Hessians at each critical point are
\begin{align*}
	\Hessian{f}(0,0) &= \begin{pmatrix} -2 & 0 \\ 0 & \frac{3}{2} \end{pmatrix} \\
	\Hessian{f}(0,-3) &= \begin{pmatrix} -2 & 0 \\ 0 & -\frac{21}{8} \end{pmatrix} \\
	\Hessian{f}(0,4) &= \begin{pmatrix} -2 & 0 \\ 0 & -\frac{7}{2} \end{pmatrix} \\
\end{align*}
So $(0,0)$ is a saddle point and the other two points are local maxima.
\end{solution}
\end{practice}

\begin{practice}% E&P Exercise 12.5.19
Classify the critical points of the function
\[
f(x,y) = 2x^2 + 8xy + y^4
\]
\begin{solution}
We have
\begin{align*}
	\pderiv{f}{x} &= 4x + 8y \\
	\pderiv{f}{y} &= 8y + 4y^3
\end{align*}
The critical points are $(0,0)$, $(4,-2)$, and $(-4,2)$.
Now
\begin{align*}
	\pderiv[2]{f}{x} &= 4 & \pderiv{f}{y,x} &= 8 \\
	\pderiv{f}{x,y} &=8    & \pderiv[2]{f}{y} &= 8 + 12y^2 \\
\end{align*}
So the Hessians at each critical point are 
\begin{align*}
	\Hessian{f}(0,0) &= \begin{pmatrix} 4 & 8 \\ 8 & 8 \end{pmatrix} \\
	\Hessian{f}(4,-2) &= \begin{pmatrix} 4 & 8 \\ 8 & 56 \end{pmatrix} \\
	\Hessian{f}(-2,4) &= \begin{pmatrix} 4 & 8 \\ 8 & 200 \end{pmatrix} \\
\end{align*}
So $(0,0)$ is a saddle point and the other two points are local minima.
\end{solution}
\end{practice}

\mode<article>{%
\section*{Guided Practice}}

\Do{Worksheet}

\mode<article>{%
\section*{Closure}}
\timecheck{\LARGE\clock{9}{50}}

The second derivative test for functions of two variables involves a complicated condition on the three derivatives.
For functions of three variables it is even more complicated.  The Hessian matrix is $3\times 3$, and the types of hyper surfaces that can happen at each critical point are more manifold.  There are two saddle-type surfaces.  If all the diagonal sub-minors have positive determinants, we know the critical point is a local minimum, but that's it.

\begin{frame}{Summary}
\begin{itemize}
\item critical points
\item second derivative test
\end{itemize}
\end{frame}



\mode<article>{%
\section*{Independent Practice}

\subsection*{Homework problems}

\begin{itemize}
\item WebAssign
\end{itemize}

}% end mode<article>


\bibliographystyle{amsalpha}
\bibliography{undergrad-textbooks}


\end{document}
%</lesson>
%<*worksheet>
\usetikzlibrary{MATH-UA-123}
\usepackage{leincalc}

\newcommand{\va}{\Vector{a}}
\newcommand{\vb}{\Vector{b}}
\newcommand{\vc}{\Vector{c}}
\newcommand{\vr}{\Vector{r}}
\newcommand{\vtau}{\Vector{\pmb{\tau}}}
\newcommand{\vzero}{\Vector{\pmb{0}}}

% Fix links with images in them.
% See http://tex.stackexchange.com/questions/31819/xelatex-includegraphics-and-hyperlink
% and http://www.tug.org/pipermail/xetex/2005-October/002480.html
\newsavebox{\ximagebox}
\newlength{\ximageheight}
\newsavebox{\xglyphbox}
\newlength{\xglyphheight}
\newcommand{\xbox}[1]%
  {\savebox{\ximagebox}{#1}%
  \settoheight{\ximageheight}{\usebox{\ximagebox}}%
  \savebox{\xglyphbox}{\char32}%
  \settoheight{\xglyphheight}{\usebox{\xglyphbox}}%
  \raisebox{\ximageheight}[0pt][0pt]{\raisebox{-\xglyphheight}[0pt][0pt]{%
    \makebox[0pt][l]{\usebox{\xglyphbox}}}}%
    \usebox{\ximagebox}%
    \raisebox{0pt}[0pt][0pt]{\makebox[0pt][r]{\usebox{\xglyphbox}}}}
\newcommand{\walphalink}[1]{\marginpar{\href{#1}{\xbox{\includegraphics{wolfram_alpha_icon.png}}}}}


\newcommand{\newVector}[2]{
	\newcommand{#1}{\Vector{#2}}
}
\newVector{\vF}{F}

\usepackage{hyperref}
%<solutions>\let\Vfill\relax
%<!solutions>\let\Vfill\vfill
%<solutions>\let\Clearpage\relax
%<!solutions>\let\Clearpage\clearpage
\usepackage{datatool}
\DTLloaddb{announcements}{../announcements.csv}
\dtlsort{endDate}{announcements}{\dtlcompare}
\begin{document}
\maketitle



\begin{problems}
\item Find the critical points of the function
\[
	f(x,y) = \frac{3}{4}y^2 + \frac{1}{24}y^3 - \frac{1}{32}y^4 -x^2
\]
\begin{solution}
We have
\begin{align*}
	\pderiv{f}{x} &= -2x  \\
	\pderiv{f}{y} &= \frac{3}{2}y + \frac{1}{8} y^2 - \frac{1}{8} y^3 \\
	&= -\frac{1}{8}y(y^2-y-12) = -\frac{1}{8}y(y+3)(y-4)
\end{align*}
Setting $\pderiv{f}{x}=0$ and $\pderiv{f}{y} = 0$ gives $x=0$, and $y=0$, $-3$, or $4$.
So there are three critical points: $(0,0)$, $(0,-3)$, $(0,4)$.
\end{solution}
%<!solutions>\vfill
\item Find the critical points of the function
\[
f(x,y) = 2x^2 + 8xy + y^4
\]
\begin{solution}
We have $\pderiv{f}{x} = 4x + 8y$ and $\pderiv{f}{y} = 8y + 4y^3$, so we must solve the system
\begin{align}
	4x + 8y &=0 \label{2a}\\
	8x + 4y^3&=0 \label{2b}
\end{align}
Isolating $x$ in \eqref{2a} gives $x=-2y$.  Substituting this into \eqref{2b} gives 
\[
- 16 y + 4y^3 = 0 \implies 4y(y-2)(y+2) = 0
\]
So $y=0$ or $y= \pm 2$.  Since $x=-2y$ we have three critical points: $(0,0)$, $(4,-2)$, and $(-4,2)$.
\end{solution}

%<!solutions>\vfill\clearpage
\item Classify the critical points of the function in Problem~1.
\begin{solution}
We have
\begin{align*}
	\pderiv{f}{x} &= -2x  \\
	\pderiv{f}{y} &= \frac{3}{2}y + \frac{1}{8} y^2 - \frac{1}{8} y^3 \\
	&= -\frac{1}{8}y(y^2-y-12) = -\frac{1}{8}y(y+3)(y-4)
\end{align*}
There are three critical points: $(0,0)$, $(0,-3)$, $(0,4)$.
Now
\begin{align*}
	\pderiv[2]{f}{x} &= -2 & \pderiv{f}{y,x} &= 0 \\
	\pderiv{f}{x,y} &=0    & \pderiv[2]{f}{y} &= \frac{3}{2} + \frac{1}{4} y - \frac{3}{8} y^2 \\
\end{align*}
The Hessians at each critical point are
\begin{align*}
	\Hessian{f}(0,0) &= \begin{pmatrix} -2 & 0 \\ 0 & \frac{3}{2} \end{pmatrix} \\
	\Hessian{f}(0,-3) &= \begin{pmatrix} -2 & 0 \\ 0 & -\frac{21}{8} \end{pmatrix} \\
	\Hessian{f}(0,4) &= \begin{pmatrix} -2 & 0 \\ 0 & -\frac{7}{2} \end{pmatrix} \\
\end{align*}
So $(0,0)$ is a saddle point and the other two points are local maxima.
Here is a contour plot showing the critical points:
\begin{center}
\includegraphics{ws3}
\end{center}
\end{solution}

%<!solutions>\vfill
\item Classify the critical points of the function in Problem~2.
\begin{solution}
Referring back to Problem~2, we have
\begin{align*}
	\pderiv[2]{f}{x} &= 4 & \pderiv{f}{y,x} &= 8 \\
	\pderiv{f}{x,y} &=8    & \pderiv[2]{f}{y} &= 12y^2 \\
\end{align*}
So the Hessians at each critical point are 
\begin{align*}
	\Hessian{f}(0,0) &= \begin{pmatrix} 4 & 8 \\ 8 & 0 \end{pmatrix} \\
	\Hessian{f}(4,-2) &= \begin{pmatrix} 4 & 8 \\ 8 & 48 \end{pmatrix} \\
	\Hessian{f}(-2,4) &= \begin{pmatrix} 4 & 8 \\ 8 & 48 \end{pmatrix} \\
\end{align*}
At $(0,0)$, the determinant of the Hessian is negative, so that critical point is a saddle point.
At the other two critical points, the determinant is positive and the top-left entry is positive, so they are local minima.
Here is a contour plot showing the critical points:
\begin{center}
\includegraphics{ws4}
\end{center}
\end{solution}
\end{problems}


%<!solutions>\vfill

\end{document}
%</worksheet>
